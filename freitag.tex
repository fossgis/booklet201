
% time: Friday 09:00
% URL: https://pretalx.com/fossgis2019/talk/N8C99J/

\noindent%
\newTimeslot{09:00}
\abstractAudimax{%
  Tobias Frechen%
}{%
  Cloudbasierte Geodateninfrastruktur\linebreak für den Glasfaserrollout in der Deutschen Telekom AG%
}{%
}{%
  Überblick über den Planungsprozess der Deutschen Telekom AG für den Glasfaserrollout mit Fokus auf
  den Aufbau und Prozesse einer Cloud-basierten Geodateninfrastruktur.%
}


%%%%%%%%%%%%%%%%%%%%%%%%%%%%%%%%%%%%%%%%%%%

% time: Friday 09:00
% URL: https://pretalx.com/fossgis2019/talk/QH7C7S/

\noindent%

\abstractMathe{%
  Hans-Jörg Stark%
}{%
  Ergebnisse der Umfrage zur QGIS-Nutzung\linebreak in der Schweiz%
}{%
}{%
  Der Beitrag stellt die Ergebnisse der QGIS-Anwendergruppe vor, die im Herbst 2018 in der Schweiz
  durchgeführt wurde.%
}


%%%%%%%%%%%%%%%%%%%%%%%%%%%%%%%%%%%%%%%%%%%

% time: Friday 09:00
% URL: https://pretalx.com/fossgis2019/talk/RE787W/

\noindent%

\abstractPhysik{%
  Daniel Koch, Marc Jansen%
}{%
  Demosession GeoServer %
}{%
}{%
  In der Demosession werden wesentliche Funktionalitäten des GeoServers live demonstriert. Dies
  umfasst die grafische Benutzer\-oberfläche (Daten einbinden, Feature-Types erzeugen, Layer\linebreak
  publizieren und stylen), die REST-Schnittstelle und ggf. zusätz\-liche Extensions.%
}


%%%%%%%%%%%%%%%%%%%%%%%%%%%%%%%%%%%%%%%%%%%

% time: Friday 09:30
% URL: https://pretalx.com/fossgis2019/talk/HZ9SJU/

\noindent%
\newTimeslot{09:30}
\abstractAudimax{%
  Arnulf Christl%
}{%
  DevOps für die GDI 4.0~-- agil stabil%
}{%
}{%
  Neue Trends und Technologien für die GDI 4.0, ein Erfahrungsbericht aus dem Einsatz von
  Open-Source-Geodaten und OpenStreetMap in der Industrie.%
}


%%%%%%%%%%%%%%%%%%%%%%%%%%%%%%%%%%%%%%%%%%%

% time: Friday 09:30
% URL: https://pretalx.com/fossgis2019/talk/QBTAWT/

\noindent%

\abstractMathe{%
  Arndt Brenschede%
}{%
  Jenseits des etablierten OSM-Routing-Regelsatzes%
}{%
}{%
  Etablierte Routing-Regeln entstehen nicht alleine durch Tagging-Proposals, sondern durch komplexe
  Wechselwirkungen, und in vielen Randbereichen auch gar nicht. Ein Überblick.%
}


%%%%%%%%%%%%%%%%%%%%%%%%%%%%%%%%%%%%%%%%%%%

% time: Friday 09:30
% URL: https://pretalx.com/fossgis2019/talk/PVVKGE/

\noindent%

\abstractPhysik{%
  Alexey Valikov%
}{%
  Von statischen Bildern zu interaktiven Karten und zurück%
}{%
}{%
  Dieser Vortrag erläutert Werkzeuge und Techniken, mit denen man statische Bilder oder PDFs schnell und
  einfach in interaktive Karten umwandeln kann.%
}


%%%%%%%%%%%%%%%%%%%%%%%%%%%%%%%%%%%%%%%%%%%

% time: Friday 10:00
% URL: https://pretalx.com/fossgis2019/talk/VPBVLS/

\noindent%
\newTimeslot{10:00}
\abstractAudimax{%
  Arne Schubert%
}{%
  GDIs in der Cloud%
}{%
}{%
  Microservice-Infrastrukturen in einer Cloud bringen viele Vorteile doch auch einige Nachteile
  gegenüber Monolithen mit sich. In dem Vortrag soll aufgezeigt werden, was zu beachten ist, damit
  eine Migration oder die neue Infrastruktur in der Cloud einen tatsächlichen Mehrwert bringt. Sei
  es was architektonisch zu beachten ist, um die Vorteile für sich nutzen zu können, oder welche
  Möglichkeiten es gibt, um nicht von den Nachteilen betroffen zu sein.%
}


%%%%%%%%%%%%%%%%%%%%%%%%%%%%%%%%%%%%%%%%%%%

% time: Friday 10:00
% URL: https://pretalx.com/fossgis2019/talk/WGCEH8/

\noindent%

\abstractMathe{%
  Christoph Hormann%
}{%
  Wenn Mapper Karten malen%
}{%
}{%
  Warum bei der Datenerfassung in OpenStreetMap manchmal\linebreak weniger mehr ist und was man als Mapper im
  Interesse von\linebreak Erfassungseffizienz, der Datenqualität und einer möglichst breiten Nützlichkeit der
  Daten beachten sollte.%
}


%%%%%%%%%%%%%%%%%%%%%%%%%%%%%%%%%%%%%%%%%%%

% time: Friday 10:00
% URL: https://pretalx.com/fossgis2019/talk/9AJUNL/

\noindent%

\abstractPhysik{%
  Jakob Miksch%
}{%
  Malawi Atlas%~-- eine SDI mit PostGIS, GeoServer und GeoExt%
}{Eine SDI mit PostGIS, GeoServer und GeoExt%
}{%
  Der Malawi Atlas ist eine Plattform um Naturgefahren in\linebreak Malawi mittels Geodaten zu visualisieren.
  Im Hintergrund wird auf bewährte Open-Source-Komponenten wie PostGIS, Geo\-Server, GeoExt und
  OpenLayers gesetzt.%
}


%%%%%%%%%%%%%%%%%%%%%%%%%%%%%%%%%%%%%%%%%%%

% time: Friday 11:00
% URL: https://pretalx.com/fossgis2019/talk/AEZM8E/

\noindent%
\newTimeslot{11:00}
\abstractAudimax{%
  Marc Jansen%
}{%
  GDI mit Docker \& Co.%
}{Einführung, Überblick und Diskussion%
}{%
  Der Talk stellt Möglichkeiten vor, um Geodateninfrastrukturen mit Hilfe von Docker zu gestalten
  und diskutiert jene.%
}


%%%%%%%%%%%%%%%%%%%%%%%%%%%%%%%%%%%%%%%%%%%

% time: Friday 11:00
% URL: https://pretalx.com/fossgis2019/talk/LK77YH/

\noindent%

\abstractMathe{%
  Harald Schwarz%
}{%
  Erfassung der Düsseldorfer Gasbeleuchtung%
}{%
}{%
  Ich möchte mein Projekt \emph{Erfassung der Düsseldorfer Gasbeleuchtung
  in OpenStreetMap} vorstellen. Ich habe seit 2010 den kompletten Bestand
  der Düsseldorfer Gasbeleuchtung (ca. 15000 Gaslaternen, 1200 gasbeleuchtete
  Straßen in OpenStreetMap erfasst. Ich möchte berichten, welche Erfahrungen
  ich bei meinen Spaziergängen durch Düsseldorf mit GPS und Fotokamera gemacht
  habe, wie die Daten in OSM eingepflegt wurden und
  wie die OSM-Daten nutzbar gemacht werden.%
}


%%%%%%%%%%%%%%%%%%%%%%%%%%%%%%%%%%%%%%%%%%%

% time: Friday 11:00
% URL: https://pretalx.com/fossgis2019/talk/RCVMHU/

\noindent%

\abstractPhysik{%
  Karsten Vennemann, Pirmin Kalberer%
}{%
  Vergleich und Benchmark der Generierung\linebreak von Karten-Vektorkacheln via MapServer\linebreak versus t-rex%
}{%
}{%
  Karten-Vektorkacheln (aka vector tiles) sind ein Datenformat das besonders für interaktive
  Web-GIS-Anwendungen interessant ist. Die Präsentation stellt zwei technische Möglichkeiten vor,
  um Vektortiles zu erstellen. Dabei wird MapServer als traditioneller Map-Rendering-Engine
  mit dem neuen Paket T-REX, einem Vektortile-Server in der Konfiguration und in einem
  Benchmark-Test verglichen.%
}


%%%%%%%%%%%%%%%%%%%%%%%%%%%%%%%%%%%%%%%%%%%

% time: Friday 11:30
% URL: https://pretalx.com/fossgis2019/talk/LU3EUC/

\noindent%
\newTimeslot{11:30}
\abstractAudimax{%
  Volker Mische%
}{%
  Einführung in dezentrale Infrastrukturen und IPFS%
}{%
}{%
  Werden Geodaten auf einem verteilten, dezentralen System gespeichert bietet dies einige Vorteile.
  Es führt zu größerer Ausfallsicherheit, besser Erreichbarkeit und mehr Sicherheit. Dieser Vortrag
  gibt einen Einblick in diese Entwicklung und zeigt anhand von IPFS, dem InterPlanetary Filesystem,
  wie dies dann im Alltag aussieht und welche Vorteile es bietet. IPFS ist Open Source unter der
  MIT-Lizenz.%
}


%%%%%%%%%%%%%%%%%%%%%%%%%%%%%%%%%%%%%%%%%%%

% time: Friday 11:30
% URL: https://pretalx.com/fossgis2019/talk/FHFRLV/

\noindent%

\abstractMathe{%
  Sarah Hoffmann%
}{%
  Im Frühtau zu Berge%
}{10 Jahre Wanderkarten mit OSM%
}{%
  Der Vortrag blickt zurück auf die Entwicklung der Wanderkarte
  waymarkedtrails.org. Zu deren zehnjährigem Bestehen
  wird vorgestellt, was sich beim Mapping für Wander-, Rad- und
  anderen Routen in OpenStreetMap getan hat.%
}


%%%%%%%%%%%%%%%%%%%%%%%%%%%%%%%%%%%%%%%%%%%

% time: Friday 11:30
% URL: https://pretalx.com/fossgis2019/talk/L77URJ/

\noindent%

\abstractPhysik{%
  Thomas Skowron%
}{%
  Vektortiles hinter den Kulissen%
}{%
}{%
  Vektortiles verdrängen an vielen Stellen Bitmaps, aber wie werden sie gemacht? Wieso setzt sich
  das MVT Format durch und was kann es? Und wurde WMS neu erfunden?%
}


%%%%%%%%%%%%%%%%%%%%%%%%%%%%%%%%%%%%%%%%%%%

% time: Friday 12:00
% URL: https://pretalx.com/fossgis2019/talk/CS9K8W/

\noindent%
\newTimeslot{12:00}
\abstractAudimax{%
  Christian Strobl%
}{%
  Das "`Cloud Optimized GeoTIFF"'%
}{Wenig Theorie und viel Praxis%
}{%
  Ein "`Cloud Optimized GeoTIFF"' ist eine normale GeoTIFF-Datei, die eine spezielle interne Struktur
  aufweist, die für spezielle HTTP-Aufrufe optimiert ist. Neben einer Vorstellung der Spezifikation
  und typischen Anwendungsbeispielen wird die Erstellung von "`Cloud Optimized GeoTIFFs"' mit GDAL
  vorgestellt. Dabei werden unterschiedliche Strategien zur Erstellung von Overviews diskutiert, die
  die Verarbeitungszeit erheblich verkleinern können.%
}


%%%%%%%%%%%%%%%%%%%%%%%%%%%%%%%%%%%%%%%%%%%

% time: Friday 12:00
% URL: https://pretalx.com/fossgis2019/talk/MDHW9L/

\noindent%

\abstractMathe{%
  Alexander Matheisen%
}{%
  Produktion generalisierter Eisenbahnkarten%
}{%
}{%
  In diesem Vortrag wird die Erstellung hochwertiger, Topologie-fähiger Eisenbahn-Streckenkarten
  auf der Basis von OpenStreet\-Map-Daten vorgestellt. Der verwendete Workflow kombiniert\linebreak
  automatisierte Prozesse mit manueller Bearbeitung und nutzt moderne, datengetriebene
  Vektortile-Technologien für die Publikation als Webkarten.%
}




%%%%%%%%%%%%%%%%%%%%%%%%%%%%%%%%%%%%%%%%%%%

% time: Friday 13:30
% URL: https://pretalx.com/fossgis2019/talk/RWNTVR/

\noindent%
\newTimeslot{13:30}
\abstractAudimax{%
  Anita Graser%
}{%
  Keynote: Einblicke vom Bazaar des QGIS-Projekts%
}{%
}{%
  Ein Blick hinter die Kulissen des QGIS-Projekts: Woher es kommt, wie es tickt und wohin es weiter
  geht.%
}


%%%%%%%%%%%%%%%%%%%%%%%%%%%%%%%%%%%%%%%%%%%

% time: Friday 14:00
% URL: https://pretalx.com/fossgis2019/talk/JTK3VZ/

\noindent%
\newTimeslot{14:00}
\abstractAudimax{%
  Dominik Helle, Frank Schwarzbach%
}{%
  Abschluss der drei Konferenztage%
}{%
}{%
  Abschluss der drei Konferenztage. Es geht nach dem Sektempfang mit OSM weiter~-- mit dem berühmten
  OSM-Quiz, Exkursion, Diskussion, Vorträgen, ...%
}


%%%%%%%%%%%%%%%%%%%%%%%%%%%%%%%%%%%%%%%%%%%

% time: Friday 14:30
% URL: https://pretalx.com/fossgis2019/talk/XTNJL7/

\noindent%
\newTimeslot{14:30}
\abstractOther{%
  Astrid Emde%
}{%
  Sektempfang%
}{%
}{%
  Der FOSSGIS-Verein lädt seine Mitglieder, Freunde, Unterstützer und Interessierte herzlich zum
  geselligen Beisammensein nach Abschluss der Konferenz an den FOSSGIS-Stand im Ausstellerbereich
  ein.%
}


%%%%%%%%%%%%%%%%%%%%%%%%%%%%%%%%%%%%%%%%%%%

% time: Friday 15:00
% URL: https://pretalx.com/fossgis2019/talk/UTEVQD/

\noindent%
\newTimeslot{15:00}
\abstractAudimax{%
  Christopher Lorenz%
}{%
  OSM-Quiz%
}{Wie gut kennst du OSM?%
}{%
  Das OSM-Quiz bietet als Fortsetzung der Events der letzten Jahre wieder spannende Fragen zu
  interessanten Fakten. Jeder ist herzlich eingeladen mitzuraten, um sein Wissen im Umfeld von
  OpenStreetMap und GIS zu testen und vielleicht auch etwas aufzufrischen.%
}


%%%%%%%%%%%%%%%%%%%%%%%%%%%%%%%%%%%%%%%%%%%

% time: Friday 15:30
% URL: https://pretalx.com/fossgis2019/talk/FAMBCJ/

\noindent%
\newTimeslot{15:30}
\abstractAudimax{%
  Marc Tobias Metten%
}{%
  Küsten, Meere, Zeitzonen%
}{Uneditierbar große Polygone%
}{%
  Wir schauen auf mehrere Datensätze, die auf OpenStreetMap\linebreak basieren aber aufgrund ihrer Größe ohne
  separate Prozesse nicht in den üblichen Editoren bearbeitet werden können. Als Beispiel erstellen
  wir ein Polygon der Eurostaaten.%
}


%%%%%%%%%%%%%%%%%%%%%%%%%%%%%%%%%%%%%%%%%%%

% time: Friday 15:30
% URL: https://pretalx.com/fossgis2019/talk/KYNAHY/

\noindent%

\abstractMathe{%
  Hartmut Holzgraefe%
}{%
  Viele Kartenstile parallel installieren%
}{%
}{%
  Verschiedene Mapnik-Kartenstile haben unterschiedliche Daten\-bank- und Shapefile-Abhängigkeiten.
  Trotzdem ist es mit ein klein wenig Aufwand möglich die meisten öffentlich verfügbaren Stile
  parallel zu installieren.%
}

%%%%%%%%%%%%%%%%%%%%%%%%%%%%%%%%%%%%%%%%%%%

% time: Friday 16:30
% URL: https://pretalx.com/fossgis2019/talk/RLJTSG/

\noindent%
\newTimeslot{16:30}
\abstractAudimax{%
  Falk Zscheile%
}{%
  Craftmapping und DSGVO%
}{Was ist erlaubt, wo sind die Grenzen?%
}{%
  Das OpenStreetMap-Projekt sammelt geographische Informationen. Dabei ergeben sich zahlreiche
  Berührungspunkte mit personenbezogenen Informationen. Der Vortrag geht den Fragen nach, ob die
  Datenschutzgrundverordnung (DSGVO) einschlägig ist und welche Anforderungen  ggf. beim Craftmapping und
  der Eintragung von Informationen in die Datenbank zu beachten sind.%
}


%%%%%%%%%%%%%%%%%%%%%%%%%%%%%%%%%%%%%%%%%%%

% time: Friday 16:30
% URL: https://pretalx.com/fossgis2019/talk/P8NJKD/

\noindent%

\abstractMathe{%
  Hanna Krüger%
}{%
  Das OSM-Wiki%
}{Die eierlegende Wollmilchsau der Community%
}{%
  Ein Rundumschlag zum OSM-Wiki: Was steht eigentlich alles drin, welche Stärken und Schwächen hat
  das Konzept und wie könnte man den Problemen des Wikis entgegenwirken.%
}


%%%%%%%%%%%%%%%%%%%%%%%%%%%%%%%%%%%%%%%%%%%

% time: Friday 17:00
% URL: https://pretalx.com/fossgis2019/talk/GP8WDX/

\noindent%
\newTimeslot{17:00}
\abstractMathe{%
  Christopher Lorenz%
}{%
  Fahrradknotenpunkte in OpenStreetMap%
}{%
}{%
  Der Vortrag zeigt, was Fahrradknotenpunkte sind und wie sie in OpenStreetMap erfasst werden. Neben
  der Darstellung der länderabhängigen Verbreitung und Erfassung der Knotenpunkte wird die
  Machbarkeit einer Auswertung der Knotenpunkte in OpenStreetMap präsentiert.%
}


%%%%%%%%%%%%%%%%%%%%%%%%%%%%%%%%%%%%%%%%%%%
