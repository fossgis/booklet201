
% time: Thursday 09:00
% URL: https://pretalx.com/fossgis2019/talk/VM37SR/

%
\newTimeslot{09:00}
\abstractAudimax{%
  Claas Leiner, Bernhard Ströbl%
}{%
  QGIS%
}{Das GIS mit unbegrenzten Darstellungsmöglichkeiten%
}{%
  Bei der Kartendarstellung im QGIS wird es immer schwieriger,
  auf wirklich unüberwindliche Grenzen zu stoßen.

  Mit Hilfe des verschachtelbaren Symbol-Layer-Konzeptes,\linebreak einiger spezieller Symbol-Layer-Typen und
  des Geometriegenerators, lassen sich auch ungewöhnliche  Flächensignaturen und\linebreak exzentrische Ideen
  umsetzen.  Regelbasierte  Darstellung und\linebreak datendefinierte  Übersteuerungen machen die Inhalte
  sämtlicher Attribute gleichzeitig für die Darstellung nutzbar. Ausnahmen werden einfach definierbar.%
}


%%%%%%%%%%%%%%%%%%%%%%%%%%%%%%%%%%%%%%%%%%%

% time: Thursday 09:00
% URL: https://pretalx.com/fossgis2019/talk/ZZ7EJH/

%

\abstractMathe{%
  Richard Figura, Alexander Willner, Michael Martin%
}{%
  Smarte Daten im Knowledge Graph%
}{Die Grundlage einer zukunftssicheren Bereitstellung\linebreak offener Daten%
}{%
  Offene Daten sind einer der wichtigsten Rohstoffe der digitalen Welt, mit wachsender
  wirtschaftlicher und gesellschaftlicher Bedeutung. Trotz zahlreicher Bemühungen konnten
  prognostizierte Mehrwerte noch nicht erreicht werden, was unter anderem auf eine unvollständige
  Vernetzung der Daten zurückzuführen ist. In diesem Vortrag werden Technologien und Prozesse
  vorgestellt, um Daten zu einem öffentlichen verfügbaren Knowledge Graph hinzuzufügen und dort mit
  Daten anderer Quellen zu verknüpfen.%
}


%%%%%%%%%%%%%%%%%%%%%%%%%%%%%%%%%%%%%%%%%%%

% time: Thursday 09:00
% URL: https://pretalx.com/fossgis2019/talk/USA8LS/

%

\abstractPhysik{%
  Doris Schuller, David Kirschheuter%
}{%
  Wie Archäologinnen GIS (nicht) nutzen%
}{%
}{%
  In der archäologischen Forschung werden GIS bereits erfolgreich eingesetzt.
  Wie sieht es im Bereich Grabungsdokumentation aus?%
}


%%%%%%%%%%%%%%%%%%%%%%%%%%%%%%%%%%%%%%%%%%%

% time: Thursday 09:30
% URL: https://pretalx.com/fossgis2019/talk/79VWTC/

%
\newSmallTimeslot{09:30}
\abstractMathe{%
  Andreas Krumtung%
}{%
  Quo vadis Open Data%
}{Geoportale von Bund und Ländern auf dem Prüfstein%
}{%
  Der Beitrag gibt einen Überblick über die Geoportallandschaft des Bundes und der Länder in
  Deutschland und zeigt auf, welche Hausaufgaben Bund und Länder noch haben, wenn sie
  funktionierende Datenökosysteme um ihre Portale herum etablieren wollen.%
}


%%%%%%%%%%%%%%%%%%%%%%%%%%%%%%%%%%%%%%%%%%%

% time: Thursday 09:30
% URL: https://pretalx.com/fossgis2019/talk/S7NKWQ/

%

\abstractPhysik{%
  Christian Trapp%
}{%
  Tachy2GIS%
}{Mit der Totalstation zeichnen%
}{%
  Tachy2GIS ist ein QGIS-Plugin zur direkten Erfassung dreidimensionaler Geometrien mit dem
  Tachymeter.%
}


%%%%%%%%%%%%%%%%%%%%%%%%%%%%%%%%%%%%%%%%%%%

% time: Thursday 10:00
% URL: https://pretalx.com/fossgis2019/talk/PLRQCU/

%
\newTimeslot{10:00}
\abstractAudimax{%
  Johannes Kröger%
}{%
  Kartografie-Rezepte\linebreak für die Experimentalküche%
}{%
}{%
  Letztes Jahr gab es "`5-Minuten-Kartografie-Rezepte aus der QGIS-Trickkiste"' als Lightning-Talk,
  ein wilder Ritt durch einige\linebreak Spielereien ohne Zeit für Erklärungen. In dieser Demo-Session werden
  wieder ähnlich interessante, ausgefallene, praktische oder künstlerische kartografische Kniffe
  gezeigt und die Herangehensweise diesmal \emph{ausführlich} erläutert.%
}

%%%%%%%%%%%%%%%%%%%%%%%%%%%%%%%%%%%%%%%%%%%

% time: Thursday 10:00
% URL: https://pretalx.com/fossgis2019/talk/9FTWCX/

%
\enlargethispage{1\baselineskip}
\abstractMathe{%
  Armin Retterath%
}{%
  Der neue Standard für Darstellungsdienste\linebreak in Deutschland%
}{%
}{%
  Im Rahmen der Veranstaltung wird der neue Standard für die\linebreak interoperable Bereitstellung von WMS-
  und WMTS-Diensten\linebreak innerhalb der Geodateninfrastruktur Deutschland (GDI-DE) vorgestellt, und es
  wird anhand praktischer Beispiele erläutert, welche Konsequenzen dies für die bereitstellenden
  Institutionen hat.%
}
%%%%%%%%%%%%%%%%%%%%%%%%%%%%%%%%%%%%%%%%%%%
\sponsorBoxA{203_dbg.jpg}{0.30\textwidth}{3}{%
\textbf{Bronzesponsor, Aussteller}\\
Die d.b.g. Datenbankgesellschaft mbH bietet Softwarelösungen und
Dienstleistungen für ein innovatives Freiraummanagement. Das Angebot reicht von
Lösungen für die Erfassung von Daten über die Planung bis zur
Betriebssteuerung. Überdies stellt die d.b.g QGIS-Know-how zur Verfügung.
}

%%%%%%%%%%%%%%%%%%%%%%%%%%%%%%%%%%%%%%%%%%%

% time: Thursday 10:00
% URL: https://pretalx.com/fossgis2019/talk/QPNPK7/

%

\abstractPhysik{%
  Marco Block-Berlitz, Hendrik Rohland, Christina Franken%
}{%
  QGIS als Forschungswerkzeug in der\linebreak Archäologie~-- Anwendungen bei der mongolisch-deutschen Orchon-Expedition%
}{%
}{%
  Im Rahmen der mongolisch-deutschen Orchon-Expedition wird QGIS als Standardwerkzeug für die
  Planung und Durchführung von Kampagnen eingesetzt. Im Besonderen soll die Befliegungskampagne im
  September 2018 vorgestellt werden. Bei dieser\linebreak wurden innerhalb von nur fünf Tagen mehr als 50
  Quadratkilometer aufgenommen und rekonstruiert. Die Herausforderung der Logistik einer
  solchen Kampagne erfordert eine akkurate\linebreak Planung im Vorfeld und vor Ort. Die wichtige Rolle von
  QGIS wird gezeigt.%
}
%%%%%%%%%%%%%%%%%%%%%%%%%%%%%%%%%%%%%%%%%%%
\vspace{2\baselineskip}

\sponsorBoxA{204_geops.pdf}{0.37\textwidth}{3}{%
\textbf{Bronzesponsor}\\
Die Verarbeitung von Geodaten für die Bereiche Umwelt und Mobilität steht im
Fokus unserer Arbeit. Die Bandbreite reicht von einfachen Web-Karten über
Geschäftslösungen bis hin zu kompletten Geodaten-Infrastrukturen. Mit der
Verarbeitung von Echzeitdaten liefert geOps Fahrgast-Informationsstyeme für
den öffentlichen Verkehr.
}

%%%%%%%%%%%%%%%%%%%%%%%%%%%%%%%%%%%%%%%%%%%

% time: Thursday 11:00
% URL: https://pretalx.com/fossgis2019/talk/877HVT/

%
\newTimeslot{11:00}
\abstractAudimax{%
  Jörg Thomsen%
}{%
  Vergleich QGIS-Server, Geoserver\linebreak und MapServer%
}{%
}{%
  Wo liegen die Unterschiede zwischen den drei Engines, was sind Gemeinsamkeiten, Schnittstellen und
  womöglich spezifische\linebreak Einsatzgebiete?%
}


%%%%%%%%%%%%%%%%%%%%%%%%%%%%%%%%%%%%%%%%%%%

% time: Thursday 11:00
% URL: https://pretalx.com/fossgis2019/talk/GY7EGG/

%

\abstractMathe{%
  Axel Lorenzen-Zabel%
}{%
  OpenGeoEdu%
}{Mit offenen Daten lernen%
}{%
  Wir stellen Ihnen den ersten offenen Online-Kurs mit dem Kern\-thema Open Data vor und geben
  Einblicke in die Ergebnisse\linebreak unseres ersten Semesters%
}


%%%%%%%%%%%%%%%%%%%%%%%%%%%%%%%%%%%%%%%%%%%

% time: Thursday 11:00
% URL: https://pretalx.com/fossgis2019/talk/CJGV7F/

%

\abstractPhysik{%
  Daniel Kastl%
}{%
  Dotloom%
}{Große Point-Cloud-Daten im Distributed Web%
}{%
  Dotloom ist ein Open-Source-Projekt und ermöglicht die Synchronisation, Replikation, Indexierung
  und Verarbeitung von Terabyte an Geodaten mit Peer-to-Peer-Technologien. Aufbauend auf dem
  "`DAT-Projekt"' erweitert Dotloom die Funktionalität speziell zur Verwaltung, Abfrage und
  Visualisierung von Point-Cloud-Daten.%
}


%%%%%%%%%%%%%%%%%%%%%%%%%%%%%%%%%%%%%%%%%%%

% time: Thursday 11:30
% URL: https://pretalx.com/fossgis2019/talk/8VEUHT/

%
\newTimeslot{11:30}
\abstractAudimax{%
  Jan Suleiman, Christian Mayer, Kai, Daniel Koch%
}{%
  GeoStyler%~-- ein generischer grafischer Stileditor für Geodaten%
}{Ein generischer grafischer Stileditor für Geodaten%
}{%
  GeoStyler ist eine react-basierte Open-Source-UI-Bibliothek zur Erstellung von Stileditoren für
  Web-GIS-Anwendungen. Der Vortrag stellt den aktuellen Stand und mögliche nächste Schritte vor.%
}


%%%%%%%%%%%%%%%%%%%%%%%%%%%%%%%%%%%%%%%%%%%

% time: Thursday 11:30
% URL: https://pretalx.com/fossgis2019/talk/NBXCC7/

%

\abstractMathe{%
  Sebastian Meier%
}{%
  Mehrwert für Bürger(innen) schaffen%
}{%
}{%
  Wie können offene räumliche Daten zum Nutzen der Zivilgesellschaft sinnvoll aufbereitet werden?
  Das Ideation \& Prototyping Lab der gemeinnützigen Technologiestiftung Berlin gibt Ein\-blicke in
  räumliche Open-Data-Anwendungen für die Bürgerinnen und Bürger Berlins.%
}


%%%%%%%%%%%%%%%%%%%%%%%%%%%%%%%%%%%%%%%%%%%

% time: Thursday 11:30
% URL: https://pretalx.com/fossgis2019/talk/E8QA8U/

%

\abstractPhysik{%
  Jelto Buurman%
}{%
  Verarbeitung von DGM-Daten\linebreak und Laserscandaten mit QGIS%
}{%
}{%
  Ausgehend von einem kurzen Abriss über Laserscandaten, wird dargestellt, wie diese Daten in QGIS
  aufbereitet und verarbeitet werden können. Es werden einige Beispiele für die Anwendung von
  Laserscandaten vorgestellt. Abschließend wird anhand einer Live-Präsentation die hervorragende Performance gezeigt.%
}


%%%%%%%%%%%%%%%%%%%%%%%%%%%%%%%%%%%%%%%%%%%

% time: Thursday 12:00
% URL: https://pretalx.com/fossgis2019/talk/8MZZGX/

%
\newTimeslot{12:00}
\abstractAudimax{%
  Oliver Fink, Johannes Lauer%
}{%
  HERE XYZ \& QGIS%~-- ein neuer Open-Source-Map-Hub made by HERE%
}{Ein neuer Open-Source-Map-Hub made by HERE%
}{%
  HERE XYZ ist eine Sammlung von Open-Source-Tools, um die\linebreak Arbeiten mit Geodaten zu vereinfachen.
  Die Basis ist der XYZ Hub. Es ist ein Echtzeit-Cloud-basierter Location Hub zum Auffinden,
  Speichern, Laden, Bearbeiten und Veröffentlichen von privaten oder öffentlichen geographischen
  Daten. Wir zeigen die Vorteile dieser Tools, insbesondere die Interoperabilität mit etablierten
  Open-Source-Lösung wie QGIS.%
}


%%%%%%%%%%%%%%%%%%%%%%%%%%%%%%%%%%%%%%%%%%%

% time: Thursday 12:00
% URL: https://pretalx.com/fossgis2019/talk/AP8QL3/

%

\abstractMathe{%
  Uwe Raudszus, Roland Olbricht, Martin Kucharzewski%
}{%
  Barrierefreies Fußgängerrouting\linebreak für Dortmund%
}{%
}{%
  Die Stadt Dortmund entwickelt für Fußgänger ein barrierefreies Routingsystem von Tür zu Tür für
  Menschen, die sehbehindert, blind, hörgeschädigt oder mobilitätseingeschränkt sind.

  Die Lösung bleibt nachhaltig durch Integration in die App des lokalen Verkehrsverbundes und für
  Dritte verfügbar durch Aufbau auf OpenStreetMap. So können durch die Koordinierung von mehreren
  Akteuren bestehende Strukturen mit geringem Aufwand in jeder Hinsicht effizienter genutzt werden.%
}


%%%%%%%%%%%%%%%%%%%%%%%%%%%%%%%%%%%%%%%%%%%

% time: Thursday 12:00
% URL: https://pretalx.com/fossgis2019/talk/BL9N8S/

%

\abstractPhysik{%
  Mira Kattwinkel%
}{%
  GRASS GIS und R zur Datenaufbereitung\linebreak für räumliche Regressionsmodelle%
}{%
}{%
  Das R-Paket openSTARS ermöglicht die Aufbereitung von Geodaten zur Erstellung räumlicher
  Regressionsmodelle und bietet so eine freie, auf R und GRASS GIS basierende Alternative zur
  ArcGIS-Toolbox STARS.%
}

%%%%%%%%%%%%%%%%%%%%%%%%%%%%%%%%%%%%%%%%%%%
\noindent\sponsorBoxA{205_omniscale}{0.42\textwidth}{3}{%
\textbf{Bronzesponsor}\\
Omniscale ist Spezialist für das Bereitstellen von schnellen und attraktiven
Karten auf Basis von OpenStreetMap und amtlichen Daten. Unter
maps.omniscale.com bietet Omniscale performante, kostengünstige Kartendienste
für Desktop- und Online-Anwendungen an. Omniscale ist zudem Initiator von
MapProxy und Imposm.%
}

\vspace{\baselineskip}

\input{sponsorentexte/206-mapwebbing.tex}

%%%%%%%%%%%%%%%%%%%%%%%%%%%%%%%%%%%%%%%%%%%

% time: Thursday 13:30
% URL: https://pretalx.com/fossgis2019/talk/9PEAWQ/

%
\newTimeslot{13:30}
\abstractAudimax{%
  Andreas Neumann%
}{%
  QGIS 3D%
}{%
}{%
  Seit QGIS 3.0 gibt es erste echte 3D-Visualisierungsmöglichkeiten im QGIS-Kern, ohne Plugins oder
  Drittsoftware installieren zu müssen. Die Präsentation zeigt, was im Bereich 3D-Visualisierung
  bereits möglich ist und wo es noch Probleme gibt.%
}


%%%%%%%%%%%%%%%%%%%%%%%%%%%%%%%%%%%%%%%%%%%

% time: Thursday 13:30
% URL: https://pretalx.com/fossgis2019/talk/AYKS9Z/

%

\abstractMathe{%
  Torsten Brassat%
}{%
  SHOGun-QGIS-Integration%~-- WebGIS-Applikationen vom Desktop administrieren%
}{WebGIS-Applikationen vom Desktop administrieren%
}{%
  Das Open-Source-QGIS-Plugin SHOGun-Editor zeigt, wie SHOGun-basierte Web-GIS-Applikationen in QGIS
  im Hinblick auf Hinzufügen und Stylen von Layern und Applikationen administriert werden können.%
}
%%%%%%%%%%%%%%%%%%%%%%%%%%%%%%%%%%%%%%%%%%%
\noindent\sponsorBoxA{208_camptocamp.png}{0.45\textwidth}{5}{%
\textbf{Bronzesponsor, Aussteller}\\
Camptocamp gehört zu den führenden Schweizer Dienstleistern im Bereich
von Open-Source-GIS und ist durch sein Engagement in diversen
Open-Source-Communitys international geschätzt. Wir stützen uns auf über
10~Jahre Erfahrung in der Umsetzung von innovativen \mbox{(Web-)GIS}-Lösungen für
Behörden und Unternehmen
}

%%%%%%%%%%%%%%%%%%%%%%%%%%%%%%%%%%%%%%%%%%%

% time: Thursday 13:30
% URL: https://pretalx.com/fossgis2019/talk/WSALG8/

%

\abstractPhysik{%
  Nikolai Janakiev%
}{%
  Data Science mit OpenStreetMap%
}{%
}{%
  Data Science ist ein populäres Schlagwort, das schon vielerlei Bereiche befallen hat, nicht
  zuletzt die Welt der Geoinformatik. Hier geht es darum, wie man gängige Methoden von Data Science
  auf OpenStreetMap-Daten mithilfe von Open-Source-Werkzeugen anwenden kann und daraus neue
  Einblicke erzeugen kann.%
}


%%%%%%%%%%%%%%%%%%%%%%%%%%%%%%%%%%%%%%%%%%%

% time: Thursday 14:00
% URL: https://pretalx.com/fossgis2019/talk/HBKJM9/

%
\newSmallTimeslot{14:00}
\abstractAudimax{%
  Marco Bernasocchi%
}{%
  QField%
}{Der mobile QGIS Alleskönner%
}{%
  QGIS is efficient and comfortable in everyday office life. However, data collection often begins
  on the field. Whether in shiver or sunshine, working outdoors requires a solution that is
  optimized for mobile devices. QField is the perfect companion of QGIS.

  In this extended demo we will show you the most important features currently available.%
}


%%%%%%%%%%%%%%%%%%%%%%%%%%%%%%%%%%%%%%%%%%%

% time: Thursday 14:00
% URL: https://pretalx.com/fossgis2019/talk/ACCU9K/

%

\abstractMathe{%
  Mathias Gröbe%
}{%
  Aktuelle Möglichkeiten der kartographischen Reliefdarstellung%
}{%
}{%
  Ideen für eine bessere Geländevisualisierung mit aktuellen Methoden und altbekanntem Wissen: Ein
  Überblick über Potenzial, Fehlerquellen und Möglichkeiten.%
}


%%%%%%%%%%%%%%%%%%%%%%%%%%%%%%%%%%%%%%%%%%%

% time: Thursday 14:00
% URL: https://pretalx.com/fossgis2019/talk/ZTRDY9/

%

\abstractPhysik{%
  Roland Olbricht%
}{%
  Wie aktuell sind OpenStreetMap-Daten?%
}{%
}{%
  Wenn die Qualität von OpenStreetMap-Daten diskutiert wird, ist eine große offene Frage, ob wir
  OpenStreetMap-Mapper die Daten dauerhaft aktuell halten können.
 In dem Vortrag wird es einerseits um Werkzeuge gehen, wie man abschätzen kann, welche Daten wohl
  wie aktuell sind, um Zweifel bei Mappern und Außenstehenden ausräumen zu können. Andererseits wird
  es um Methodiken gehen, wie man das Aktuell-Halten bequem und\linebreak attraktiv machen kann.%
}


%%%%%%%%%%%%%%%%%%%%%%%%%%%%%%%%%%%%%%%%%%%

% time: Thursday 14:30
% URL: https://pretalx.com/fossgis2019/talk/X8VMWG/

%
\newSmallTimeslot{14:30}
\abstractAudimax{%
  Numa Gremling%
}{%
  Leaflet%
}{Komfortabel Web-Maps erstellen%
}{%
  Leaflet ist eine der momentan am häufigsten benutzten und\linebreak beliebtesten
  Bibliotheken um Web-Maps zu erstellen. In diesem Vortrag lernen Sie, wieso.%
}
%%%%%%%%%%%%%%%%%%%%%%%%%%%%%%%%%%%%%%%%%%%
\noindent\sponsorBoxA{207_graphhopper.png}{0.42\textwidth}{2}{%
\textbf{Bronzesponsor, Aussteller}\\
Die GraphHopper GmbH entwickelt Werkzeuge zur Routen- und Tourenplanung unter
Einsatz von Open Data wie OpenStreetMap. Große Teile der eigenen Software, wie
die GraphHopper Routing Engine und die jsprit-Bibliothek, werden als Open-Source-Software
veröffentlicht.
}

%%%%%%%%%%%%%%%%%%%%%%%%%%%%%%%%%%%%%%%%%%%

% time: Thursday 14:30
% URL: https://pretalx.com/fossgis2019/talk/TLJ8AX/

%

\abstractPhysik{%
  Jochen Topf%
}{%
  Osmoscope%
}{Ein neues QA-Tool für OpenStreetMap%
}{%
  Osmoscope ist ein neues Tool zur Qualitätssicherung von OSM-Daten. Aufbereitung der Daten und
  Webclient sind komplett\linebreak getrennt. Jeder kann einfach eigene Layer im Web veröffentlichen und
  Mapper können sie per Mausklick in Osmoscope einbinden.%
}

%%%%%%%%%%%%%%%%%%%%%%%%%%%%%%%%%%%%%%%%%%%

% time: Thursday 15:30
% URL: https://pretalx.com/fossgis2019/talk/KRBJGJ/

%
\newSmallTimeslot{15:30}
\abstractAudimax{%
  Marc Jansen, Andreas Hocevar%
}{%
  OpenLayers%
}{Stand und aktuelle Entwicklungen%
}{%
  Im Vortrag wird der aktuelle Stand und die potenzielle künftige Weiterentwicklung der weitverbreiteten
  JavaScript-Bibliothek OpenLayers vorgestellt.%
}
%%%%%%%%%%%%%%%%%%%%%%%%%%%%%%%%%%%%%%%%%%%
\sponsorBoxA{209_latlon.png}{0.42\textwidth}{6}{%
\textbf{Bronzesponsor}\\
lat/lon ist seit 2000 als Be\-ra\-tungs- und Softwareunternehmen aktiv und verbindet die Geschäftsfelder
Geodateninfrastruktur (GDI) und Geo"=IT"=Standards. lat/lon ist Mitglied des Open Geospatial
Consortium (OGC) und engagiert sich im Bereich der Konformitätstests. lat/lon unterstützt aktiv das
OSGeo"=Projekt deegree.%
}

%%%%%%%%%%%%%%%%%%%%%%%%%%%%%%%%%%%%%%%%%%%

% time: Thursday 15:30
% URL: https://pretalx.com/fossgis2019/talk/LDUZG3/

%

\abstractMathe{%
  Michael Reichert%
}{%
  Wenn Firmen mappen%
}{%
}{%
  In diesem Vortrag berichtet der Autor über seine Erfahrungen mit Kunden,
  die Schulungen zum Beitragen und Verbessern von OpenStreetMap-Daten in
  Anspruch genommen haben.
%
  Welche kulturellen Unterschiede bestehen zwischen einem offenen Projekt wie
  OpenStreetMap und der geschäftlichen Welt? Was sollte bei der Regulierung
  kommerziell motivierter Datenerfassung berücksichtigt werden? Welche
  Potenziale bietet eine Integration "`kommerzieller"
  Beitragender für OSM in Deutschland?%
}


%%%%%%%%%%%%%%%%%%%%%%%%%%%%%%%%%%%%%%%%%%%

% time: Thursday 15:30
% URL: https://pretalx.com/fossgis2019/talk/VZF97U/

%

\abstractPhysik{%
  Peter Neubauer%
}{%
  FOSS- und GIS-Integrationen mit Mapillary%
}{%
}{%
  In diesem Vortrag wird auf die Daten-Pipeline des Mapillary-\linebreak Projektes eingegangen. Es werden auch
  die unterschiedlichen FOSS-Komponenten beleuchtet (OpenSfM, MapillaryJS, iOS-SDK, Python-Tools,
  JOSM/iD-Integrationen).%
}

%%%%%%%%%%%%%%%%%%%%%%%%%%%%%%%%%%%%%%%%%%%

% time: Thursday 16:00
% URL: https://pretalx.com/fossgis2019/talk/EM3QFE/

%
\newSmallTimeslot{16:00}
\abstractAudimax{%
  Daniel Koch, Marc Jansen%
}{%
  3D-Geoapplikationen im Browser%
}{Überblick und Erfahrungen%
}{%
  Web-basierte Open-Source-3D-Applikationen mit geografischem Bezug sind bereits seit vielen Jahren
  technisch möglich, wie\linebreak einige Projektlösungen zeigen. Der Vortrag wird einige solcher Lösungen
  vorstellen und auch auf künftige Entwicklungen, neue Ansätze und Bibliotheken eingehen.%
}


%%%%%%%%%%%%%%%%%%%%%%%%%%%%%%%%%%%%%%%%%%%

% time: Thursday 16:00
% URL: https://pretalx.com/fossgis2019/talk/TCLACK/

%

\abstractMathe{%
  Pascal Neis%
}{%
  Untersuchung zum bezahlten und\linebreak organisierten Mapping im OSM-Projekt%~-- Zahlen und Fakten?%
}{Zahlen und Fakten%
}{%
  In diesem Vortrag wird ein prototypischer Ansatz präsentiert, wie möglicherweise bezahlte oder
  organisierte Mapper im OSM-Projekt erkannt werden können. Der zweite Teil des Vortrags widmet sich
  der Untersuchung der Beitragenden.%
}


%%%%%%%%%%%%%%%%%%%%%%%%%%%%%%%%%%%%%%%%%%%

% time: Thursday 16:00
% URL: https://pretalx.com/fossgis2019/talk/H7399S/

%

\abstractPhysik{%
  Robert Klemm%
}{%
  Drohnenbilder im WebGIS%~-- Wie kommen Drohnen-Bilddaten mithilfe von OpenDroneMap ins WebGIS?%
}{Wie kommen Drohnen-Bilddaten mithilfe von OpenDroneMap ins WebGIS?%
}{%
  Im Vortrag möchte ich zeigen, wie man in wenigen Schritten aus den Drohnen-Rohbilddaten ein
  nutzbares WebGIS-Projekt erstellen kann. Dabei wird zusammenfassend auf das OpenDroneMap-Projekt (ODM)
  eingegangen und welche Tools wichtig sind um ein WebGIS zu installieren und zu benutzen.%
}


%%%%%%%%%%%%%%%%%%%%%%%%%%%%%%%%%%%%%%%%%%%

% time: Thursday 16:30
% URL: https://pretalx.com/fossgis2019/talk/EZGELZ/

%
\newTimeslot{16:30}
\abstractAudimax{%
  Pirmin Kalberer%
}{%
  QWC2-Viewer für QGIS-Server mit Microservice-Architektur%
}{%
}{%
  Der QGIS-Webclient 2 (QWC2) ist ein moderner Kartenclient, der auf die Publikation von Karten mit
  QGIS-Server spezialisiert ist. Dank des Einsatzes von Microservices ist er sowohl für die\linebreak Erstellung
  einfacher In-House Clients als auch für umfangreiche Lösungen in Enterprise-Infrastrukturen
  geeignet.%
}


%%%%%%%%%%%%%%%%%%%%%%%%%%%%%%%%%%%%%%%%%%%

% time: Thursday 16:30
% URL: https://pretalx.com/fossgis2019/talk/T83MYX/

%

\abstractMathe{%
  Frederik Ramm%
}{%
  OpenStreetMap-Vandalismus für Datennutzer%~-- Arten, Häufigkeit, Schutzstrategien%
}{Arten, Häufigkeit, Schutzstrategien%
}{%
  Viele OpenStreetMap-Nutzer sind erstaunt, wenn sie hören, dass jede(r) einfach alles ändern kann.
  Geht da nicht ständig etwas kaputt? Treiben da nicht Teenager ihren Schabernack mit den heiligen
  Geodaten? Dieser Vortrag analysiert die Risiken und gibt Handlungsempfehlungen.%
}


%%%%%%%%%%%%%%%%%%%%%%%%%%%%%%%%%%%%%%%%%%%

% time: Thursday 16:30
% URL: https://pretalx.com/fossgis2019/talk/AAS7PH/

%

\abstractPhysik{%
  Till Adams%
}{%
  Vom Luftbild zur Trassenplanung%
}{%
}{%
  Im Vortrag zeigen wir die Verarbeitung der Open Data des\linebreak Bundeslands NRW für die Planung von
  potenziellen Trassen über Kostenoberflächen mittels KI-basierter Luftbildauswertung bis hin zum
  Web-GIS für Planer.%
}


%%%%%%%%%%%%%%%%%%%%%%%%%%%%%%%%%%%%%%%%%%%

% time: Thursday 17:00
% URL: https://pretalx.com/fossgis2019/talk/ZVFKPQ/

%
\newTimeslot{17:00}
\abstractAudimax{%
  Astrid Emde%
}{%
  Mapbender%
}{Neues aus dem Projekt%
}{%
  Mapbender ist eine Software zur einfachen Erstellung von WebGIS-Anwendungen. Über ein paar Klicks
  können mit dem webbasierten Administrations-Backend individuelle Anwendungen erstellt werden, eine
  Benutzer- und Gruppenverwaltung mit der Möglichkeit Rechte zuzuweisen.


  Der Vortrag geht vor allem auf die neuen Komponenten in Mapbender ein und stellt diese vor.
  Außerdem soll auf die Neuerungen in der Software eingegangen werden.%
}

%%%%%%%%%%%%%%%%%%%%%%%%%%%%%%%%%%%%%%%%%%%

% time: Thursday 17:00
% URL: https://pretalx.com/fossgis2019/talk/TLKKGE/

%

\abstractPhysik{%
  Christian Strobl%
}{%
  Download / Schnittstellen für Copernicus-Daten mit CODE-DE%
}{%
}{%
  Actinia ist eine Cloud-optimierte Open-Source-REST-Schnittstelle zum weithin bekannten GRASS GIS.
  In diesem Vortrag wird anhand von praktischen Beispielen gezeigt, wie freie Daten des
  Erdbeobachtungsprogramms Copernicus mit Actinia in Wert gesetzt werden.%
}

%%%%%%%%%%%%%%%%%%%%%%%%%%%%%%%%%%%%%%%%%%%

% time: Thursday 17:30
% URL: https://pretalx.com/fossgis2019/talk/YH8XUW/

%
\newTimeslot{17:30}
\abstractAudimax{%
  Jörg Thomsen%
}{%
  Mapbender Anwendertreffen%
}{%
}{%
  Zu diesem Treffen sind Anwender und Entwickler der WebGIS Client Suite Mapbender eingeladen. Aber
  es sind auch alle anderen Interessierten willkommen!%
}


%%%%%%%%%%%%%%%%%%%%%%%%%%%%%%%%%%%%%%%%%%%

% time: Thursday 17:30
% URL: https://pretalx.com/fossgis2019/talk/7Z3TNX/

%

\label{bof-donnerstag}
\abstractRecht{%
  Torsten Friebe, Dirk Stenger%
}{%
  deegree für INSPIRE Anwendertreffen%
}{%
}{%
  Zum Anwendertreffen sind Anwender und Entwickler herzlich eingeladen, die INSPIRE Netzwerkdienste
  mit dem OSGeo Projekt deegree bereits umsetzen oder dieses für die
  Zukunft planen.%
}


%%%%%%%%%%%%%%%%%%%%%%%%%%%%%%%%%%%%%%%%%%%

% time: Thursday 18:00
% URL: https://pretalx.com/fossgis2019/talk/EFGTTB/

%
\newSmallTimeslot{18:00}
\abstractMathe{%
  Dominik Helle%
}{%
  Mitgliederversammlung FOSSGIS e.V.%
}{%
}{%
  Zur jährlich stattfindenden Versammlung des FOSSGIS e.V.  sind alle Mitglieder herzlich
  eingeladen, teilzunehmen und sich zu beteiligen. Einige Themen stehen auf der Agenda. Wir laden
  ein zum Kennenlernen, zur Diskussion, Abstimmung und Neuwahlen.%
}


%%%%%%%%%%%%%%%%%%%%%%%%%%%%%%%%%%%%%%%%%%%
\input{sponsorentexte/401-tib.tex}
\vspace{2\baselineskip}

\sponsorBoxA{211_sourcepole.png}{0.4\textwidth}{4}{%
\textbf{Bronzesponsor, Aussteller}\\
Sourcepole entwickelt kundenspezifische Lösungen im Bereich Geoinformatik auf
der Basis von Open"=Source"=Software"=Komponenten. Das ganze Spektrum von der
QGIS-Kernentwicklung bis zu kompletten Web"=GIS"=Lösungen mit passenden
Support-Dienstleistungen wird aus einer Hand abgedeckt.%
}

